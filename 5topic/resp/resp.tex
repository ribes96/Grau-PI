\documentclass[a4paper]{article}
\usepackage[utf8]{inputenc}
\usepackage[spanish]{babel}

\author{Albert Ribes}
\title{Topic 5. Inter-domain routing (BGP)}

\begin{document}
    \maketitle

    \begin{enumerate}
        \item Explica qué es una política de encaminamiento y como se implementa

        {\bfseries
            Es una técnica para definir el conjunto de sistemas autónomos que se deben atravesar para hacer llegar un paquete a su destino.
            Se implementa usando BGP, un protocolo que tiene en cuenta las relaciones de peering y el contrato que se haya definido entre los pares de sistemas autónomos conectados.
        }
        \item Explica como escala la tabla de encaminamiento BGP en función de la cantidad de AS’s a
        los que está conectado un AS.


        {\bfseries
            Cada router BGP está obligado a mantener en una base de datos con TODA la información que recibe de otros routers BGP, pero únicamente puede transmitir aquella información que él va a usar. De este modo, aunque un router conozca varias rutas para llegar a un destino únicamente transmitirá una de ellas. De este modo se consigue encaminar en un rango muy grande, pues solo se transmiten las mejores rutas y cada router tiene únicamente una visión parcial de todo el sistema.
        }
        \item ¿Pará que sirve definir una dirección de loopback en un router? ¿Qué tipo de dirección
        es?

        {\bfseries
            Sirve para dar más seguridad al sistema. Puesto que la dirección de loopback se la asigna cada router a sí mismo, no hay peligro de que en el caso de que caiga un enlace esta dirección se pierda o que pueda cambiar. Si se usara una dirección normal se presentarían más problemas en caso de falla del sistema.

            Se usan direcciones ip privadas.
        }
        \item ¿Cómo resuelve BGP el problema de los blucles?

        {\bfseries
            Lo resuelve mediante el uso de dos protocolos distintos, uno de gestión (I-BGP) y otro de anunciación de rutas (E-BGP).

            I-BGP no puede generar bucles porque no retransmite las rutas aprendidas mediante I-BGP. E-BGP los evita descartando aquellos mensajes que contengan el propio sistema autónomo dentro del path-vector.
        }
        \item ¿Qué diferencia hay entre IBGP e EBGP?

        {\bfseries
            I-BGP solo sirve para interconectar los routers BGP que pertenecen a un mismo sistema autónomo. Lo utiliza el AS para gestionar sus rutas.

            E-BGP, en cambio, conecta routers que están en distintos AS, y se usa principalmente para transmitir las rutas de interdomain routing.
        }
        \item ¿Qué diferencia hay entre las redes que anuncia OSPF y las que anuncia BGP (e.g. con el
        comando network)?

        {\bfseries
            OSPF únicamente anuncia redes que están dentro del propio sistema autónomo, y se supone que únicamente se anuncian dentro del propio sistema autónomo.

            BGP se usa para anunciar redes que deben ser accedidas desde otros sistemas autónomos, y únicamente se anuncian a routers BGP.
        }

        \item Explica la diferencia entre un atributo BGP conocido (“well-known”) y otro opcional. Idem
        si el atributo es mandatorio y discrecional. Menciona algún atributo que tenga la característica de
        ser conocido y discrecional, otro que sea conocido y mandatorio y otro que sea opcional y
        transitivo.

        {\bfseries
            Un atributo ``well-known'' debe ser soportado por todos los routers BGP, mientras que los opcionales, como su nombre indica, es opcional que sean soportados.

            Si un atributo es mandatorio éste debe aparecer por fuerza en todos los mensajes update de BGP, mientras que si es discrecional es opcional que aparezca.

            Un atributo conocido y discrecional es el local preference.

            Un atributo conocido y mandatorio es el path vector.

            Un atributo opcional y transitivo es el cumunity.
        }
        \item ¿Qué significa que en una tabla BGP aparezca el atributo ORIGEN como incompleto? ¿Qué
        acción ha ejecutado el administrador del sistema para que aparezca como incompleto? ¿Qué
        efectos tiene dicha acción?

        {\bfseries
            Significa que esa ruta no se ha aprendido por BGP ni por EGP.

            El administrador del sistema ha configurado el router para que acepte rutas de otros protocolos (por ejemplo OSPF) o ha añadido manualmente una ruta.

            Con esto se consigue que el router transmita a través de BGP rutas que no se han conocido mediante BGP.
        }
        \item ¿Qué relación hay entre los atributos ATOMIC AGGREGATE y AGGREGATOR?

        {\bfseries
            El ATOMIC AGGREGATE indica se ha perdido información debido al AGGREGATOR, que indica que las subredes que se envían están agregadas.
        }
        \item Qué diferencia hay entre una política BGP inbound y una outbound. Qué atributo
        BGP te permite generar una política outbound?

        {\bfseries
            La política inbound sirve para decidir qué rutas
        }
        \item ¿Qué es una política de “AS-path-prepending?. Explica mediante un ejemplo
        sencillo como un ISP puede usar esta política. ¿Qué atributo BGP permite definir a un ISP una
        política de tráfico de tipo “outbound”? Explica mediante un ejemplo sencillo como un ISP puede
        usar esta política.

        {\bfseries

        }
        \item Explica la diferencia entre una comunidad “NO-EXPORT” y una comunidad “NOADVERTISE”. Pon un ejemplo de uso de cada una de ellas.

        {\bfseries
            Una comunidad ``NO-EXPORT'' indica que las rutas anunciadas no deben ser anunciadas a otros sistemas autónomos.

            Una comunidad ``NO-ADVERTISE'' indica que las rutas anunciadas no deben ser anunciadas a ningún otro router, ya sea del mismo AS o de otro.
        }
        \item ¿Qué diferencia hay entre asignar un “route-map” con el comando neighbor en
        modo “in” o en modo “out”? ¿Qué efectos tienen ambas acciones sobre la tabla BGP?

        {\bfseries
            
        }
        \item

        {\bfseries

        }
        \item

        {\bfseries

        }
        \item

        {\bfseries

        }
        \item

        {\bfseries

        }
        \item

        {\bfseries

        }
        \item

        {\bfseries

        }
    \end{enumerate}
\end{document}
