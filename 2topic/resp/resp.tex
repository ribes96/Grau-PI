\documentclass[a4paper,10pt]{article}
\usepackage[utf8]{inputenc}
\usepackage[spanish]{babel}
\usepackage{color}

\newcommand{\red}[1]{\textcolor{red}{#1}}

%opening
\title{Preguntas del tema 2}
\author{Albert Ribes}

\begin{document}

\maketitle

\begin{abstract}

\end{abstract}

% \section{}
\begin{itemize}
  %1
  \item \textbf{
  Explica los conceptos de servicio, arquitectura orientada a objetos y sistema distribuido.
  }

\red{Un servicio es un mecanismo para acceder a una o más competencias, donde el acceso se provee usando una interfaz prescrita y se ejerce de forma consistente con las cláusulas y las reglas especificadas en la descripción del servicio}

\red{La arquitectura orientada a servicios (SOA) es un paradigma para organizar y utilizar competencias distribuidas que pueden estar bajo el control de distintos dominios, que son miembros}

\red{Un sistema distribuido son diversos agentes de software que han de trabajar juntos para realizar algunas tareas. No tienen por qué operar en el mismo ambiente de proceso, de modo que se han de comunicar mediante una pila de protocolos sobre la red}

  %2
  \item \textbf{
  Explica brevemente el rol del protocolo WSDL (Web Services Description Language) en un
  servicio Web y como se estructura un servicio web con WSDL.
  }
  %3
  \item \textbf{
  Explica el rol de SOAP (Simple Object Access Protocol) en un servicio Web y describe el
  tipo de encapsulamiento que usa. ¿Qué diferencias hay entre un Web Service basado en JSON y
  XML?
  }
  %4
  \item \textbf{
  Explica que diferencias hay en usar RPC, RPC-XML y SOAP.
  }
  %5
  \item \textbf{
  Dibuja un esquema que defina la arquitectura de referencia de los Web Services y explica
  su funcionamiento básico.
  }
  %6
  \item \textbf{
  Explica que diferencias hay entre usar una arquitectura basada en XML o en JSON.
  }
  %7
  \item \textbf{
  Explica el funcionamiento básico de la criptografía de llave simétrica y como se asegura el
  intercambio de la llave.
  }
  %8
  \item \textbf{
  Explica el funcionamiento básico de la criptografía de llave pública. Explica en un ejemplo
  como podemos firmar un documento y a continuación encriptarlo.
  }
  %9
  \item \textbf{
  Explica cómo se puede firmar digitalmente con criptografía de llave pública.
  }
  %10
  \item \textbf{
  Explica mediante un ejemplo como funciona un KDC (Key Distribution Center) en
  la criptografía simétrica.
  }
  %11
  \item \textbf{
  Explica para qué sirve y cuál es el mecanismo de funcionamiento de una
  Autoridad de Certificación (CA), indicando si se usa con criptografía de llave simétrica o
  asimétrica.
  }
  %12
  \item \textbf{
  Explica la diferencia de funcionamiento entre un KDC (Key Distribution Center) y
  una Autoridad de Certificación (CA) y con qué tipo de criptografía se usa cada uno de ellos.
  }
\end{itemize}

\end{document}
